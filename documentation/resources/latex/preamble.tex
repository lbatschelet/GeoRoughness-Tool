
\usepackage{multicol}
\usepackage{amsmath, amsfonts, amssymb}
\usepackage{ifthen}
\usepackage{lipsum}


\usepackage{authblk}

% Graphische Pakete
\usepackage{graphicx}
\usepackage{float}
\usepackage{caption}
\usepackage{calc}
\usepackage{array}
	\newcolumntype{P}[1]{>{\raggedright\arraybackslash}p{#1}}
	\newcolumntype{R}[1]{>{\raggedleft\arraybackslash}p{#1}}
	\newcolumntype{C}[1]{>{\centering\arraybackslash}p{#1}}


% Listenanpassungen
\usepackage{enumitem}
	\setlist{nosep, leftmargin=*} % Reduziert den Einzug in Listen und Abstände zwischen Einträgen

% Mathe-Erweiterungen
\usepackage{calrsfs}
	\DeclareMathAlphabet{\pazocal}{OMS}{zplm}{m}{n}
\usepackage{mathtools}
\usepackage{booktabs}
\usepackage{colonequals}
\usepackage{siunitx}
	\DeclareSIUnit\century{century}
	\DeclareSIUnit\year{y}
\usepackage[version=4]{mhchem}
\usepackage{empheq}
\usepackage{bm}


% Code und Syntax-Highlighting
\usepackage[outputdir=../out]{minted}
\usepackage{shellesc} % Required for using minted package

	% Pfad-Management
	\newcommand{\resourcesdir}{../resources} % Default path
	\newcommand{\imgdir}{\resourcesdir/images} % Default image path

	% Pfadanpassung, wenn in GitHub Actions ausgeführt
	\ifdefined\RESOURCESDIR
	\renewcommand{\resourcesdir}{\RESOURCESDIR}
	\renewcommand{\imgdir}{\RESOURCESDIR/images}
	\fi


% Anpassungen für deutsche Sprache
\addto\captionsngerman{
	\renewcommand{\figurename}{Abb.}
	\renewcommand{\listfigurename}{Abbildungsverzeichnis}
	\renewcommand{\tablename}{Tab.}
	\renewcommand{\listtablename}{Tabellenverzeichnis}
	\renewcommand{\refname}{Literaturverzeichnis}
}


% -------------------------------------------------------------------------------------------------------------------- %
% Captions
	\captionsetup{
		justification=raggedright,
		singlelinecheck=false,
		font={small,rm}, % Schriftart der Captions auf Times setzen
		labelfont={bf}
	}

% -------------------------------------------------------------------------------------------------------------------- %




% -------------------------------------------------------------------------------------------------------------------- %
% Section counter behavior
\usepackage{chngcntr} % Load the package for changing counter behavior

	\counterwithin{figure}{section}   % Figures numbered within sections
	\counterwithin{table}{section}    % Tables numbered within sections
	\counterwithin{equation}{section} % Equations numbered within sections

	% (Optional) For numbering within subsections:
	% \counterwithin{figure}{subsection}  % Figures numbered within subsections
	% \counterwithin{table}{subsection}   % Tables numbered within subsections

% Raumgestaltung um Sektionen
\usepackage{xcolor, cancel}
%	\setlength{\parindent}{0pt}
%	\setlength{\parskip}{0pt}
%	\renewcommand{\baselinestretch}{0.8}
%

% -------------------------------------------------------------------------------------------------------------------- %
% Fonts and Colors
\usepackage{fontspec}
\usepackage[T1]{fontenc}  % T1 Font Encoding für ältere Pakete, falls erforderlich
\usepackage{titlesec}  % Paket zum Anpassen der Überschriften

	% Hauptschriftarten festlegen
	\setmainfont{Helvetica}  % Setzt die Hauptschriftart auf Times New Roman
	\setsansfont{Helvetica}  % Setzt die serifenlose Schriftart auf Helvetica

	\setmonofont{JetBrains Mono}[
		Path=../resources/fonts/JetBrainsMono/,
		Scale=0.85,
		Extension = .ttf,
		UprightFont=*-Regular,
		BoldFont=*-Bold,
		ItalicFont=*-Italic,
		BoldItalicFont=*-BoldItalic
	]

% Farbe für den Haupttext festlegen
\usepackage{etoolbox}
	\definecolor{darkgray}{gray}{0.1}
	\AtBeginDocument{\color{darkgray}}

% Schriftart für die Überschriften ändern mit titlesec
	\titleformat{\section}{\normalfont\sffamily\Large\bfseries\color{black}}{\thesection}{1em}{}
	\titleformat{\subsection}{\normalfont\sffamily\large\bfseries\color{black}}{\thesubsection}{1em}{}
	\titleformat{\subsubsection}{\normalfont\sffamily\normalsize\bfseries\color{black}}{\thesubsubsection}{1em}{}

	% Schriftart für den Titel ändern
	%! suppress = Makeatletter
	\makeatletter
	\renewcommand{\maketitle}{
		\begin{center}
		{\sffamily\LARGE\bfseries\@title\par}
			\vskip 1em
				{\large \@author}
			\vskip 1em
				{\@date}
		\end{center}
	}
	%! suppress = Makeatletter
	\makeatother

\usepackage{epigraph}

% -------------------------------------------------------------------------------------------------------------------- %
	% Makros für den Titel
	%! suppress = Makeatletter
	\makeatletter
	\newcommand{\inserttitle}{\@title}
	%! suppress = Makeatletter
	\makeatother

	%! suppress = Makeatletter
	\makeatletter
	\newcommand{\insertauthor}{\@author}
	%! suppress = Makeatletter
	\makeatother

% -------------------------------------------------------------------------------------------------------------------- %






% -------------------------------------------------------------------------------------------------------------------- %
% Bibliographie und Zitate
\usepackage{csquotes}

% Hyperref und Linkfarben
\usepackage[hidelinks]{hyperref}
	\pdfstringdefDisableCommands{
		\def\\{} % example of redefining "\\" for bookmarks
		\def\mhchem#1{#1} % Handling \mhchem to avoid errors in bookmarks
	}

	% \hypersetup{
	% 	colorlinks=true,
	% 	linkcolor=.,   % Farbe für interne Links
	% 	citecolor=.,   % Farbe für Zitate
	% 	filecolor=.,   % Farbe für Links zu Dateien
	% 	urlcolor=.     % Farbe für externe Links
	% }
	% Enhanced referencing with cleveref
\usepackage{cleveref}
% -------------------------------------------------------------------------------------------------------------------- %





% -------------------------------------------------------------------------------------------------------------------- %
% Header und Footer
\usepackage{fancyhdr} % Laden des Pakets für Kopf- und Fußzeilen

	\pagestyle{fancy} % Verwenden des "fancy" Stils
	\fancyhf{} % Alle Kopf- und Fußzeilenfelder bereinigen
	\renewcommand{\headrulewidth}{0pt} % Keine Linie im Kopfbereich
	\renewcommand{\footrulewidth}{0.4pt} % Dünne Linie im Fußbereich

	% Kopfzeile
	\fancyhead[L]{\nouppercase{\leftmark}}
	\fancyhead[C]{}
	\fancyhead[R]{Stefanie Röthlisberger, Lukas Batschelet, Florian Mohaupt}

	% Fußzeile
	\fancyfoot[L]{}
	\fancyfoot[C]{}
	\fancyfoot[R]{\thepage}

% License
\usepackage[
	type={CC},
	modifier={by-nc-sa},
	version={4.0},
]{doclicense}

\usepackage{fontawesome5}

	%! suppress = Makeatletter
	\makeatletter
	\newcommand{\github}[1]{%
		\href{#1}{\faGithubSquare}%
	}
	%! suppress = Makeatletter
	\makeatother



% -------------------------------------------------------------------------------------------------------------------- %
% Diverse Makros

	% Defina a logic equal symbol
	\newcommand{\logeq}{\ratio\Leftrightarrow}

	% Cancel out terms in equations
	\newcommand\hcancel[2][black]{\setbox0=\hbox{$#2$}%
		\rlap{\raisebox{.45\ht0}{\textcolor{#1}{\rule{\wd0}{1pt}}}}#2}

	% Outer Join symbol
	\newcommand{\ojoin}{\setbox0=\hbox{$\bowtie$}%
		\rule[-.02ex]{.25em}{.4pt}\llap{\rule[\ht0]{.25em}{.4pt}}}
		\newcommand{\leftouterjoin}{\mathbin{\ojoin\mkern-5.8mu\bowtie}}
		\newcommand{\rightouterjoin}{\mathbin{\bowtie\mkern-5.8mu\ojoin}}
		\newcommand{\fullouterjoin}{\mathbin{\ojoin\mkern-5.8mu\bowtie\mkern-5.8mu\ojoin}}
% -------------------------------------------------------------------------------------------------------------------- %





